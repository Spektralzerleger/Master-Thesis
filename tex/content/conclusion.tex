\chapter{Conclusion and Outlook}
\label{chapter:conclusion}

In this Thesis, we calculated spectral functions of strongly interacting Fermi gases directly in real frequencies without the need of numerical reconstruction methods by iteratively solving the corresponding Dyson-Schwinger equations. 

In the first part, we focused on the spectral properties of the spin-balanced BCS-BEC crossover. After introducing the basic theoretical foundations in Chapter~\ref{chapter:functional-methods} and~\ref{chapter:ultracold-gases}, we presented the main achievements of this work in Chapter~\ref{chapter:bcs-bec-crossover}. A fully selfconsistent numerical framework in real frequencies was developed. Additionally, analytic results for the non-selfconsistent boson self-energy with general mass and spin-imbalance at zero and finite temperature were derived. Novel results for the bosonic dimer and the fermionic single-particle spectral functions in the normal phase of the BCS-BEC crossover phase diagram were obtained. To benchmark our calculations, we applied our method to the unitary Fermi gas and compared with existing theoretical predictions and experimental data. Excellent agreement with previous work in the normal phase is found. The symmetry-broken phase turns out to be more challenging and requires additional work in the future.

In the second part, we applied our selfconsistent real-time framework to the Fermi polaron problem in Chapter~\ref{chapter:fermi-polaron}. In this case, some simplifications could be made which reduced the number of integrals and improved the numerical performance. Previous fRG and T-matrix results were confirmed and extended. Despite employing a selfconsistent approach, the experimental data could not be described accurately. The reason for this might involve the lack of many-body correlations, which can be investigated with the spectral approach.

The results obtained in this work promise a wide range of possible applications, including transport properties and the ab-initio calculation of spectral functions in the superfluid phase of the BCS-BEC crossover. Moreover, the numerical framework can be extended to full momentum-dependent vertices, or general mass and spin-imbalance. The spectral approach can be used to study various quasiparticle properties of strongly interacting Fermi gases at finite temperature. Also more involved processes like three-body scattering diagrams can be considered in the future.

\cleardoublepage